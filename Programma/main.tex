\documentclass[a4paper,11pt]{article}
\usepackage[italian]{babel}
\usepackage[utf8]{inputenc}
\usepackage[top=1 cm, bottom=1 cm, left=1 cm, right=1 cm]{geometry}
\usepackage{eurosym}
\usepackage{graphicx}
\usepackage{latexsym}

\graphicspath{{../Loghi/}}

\begin{document}
	\pagestyle{empty}
	\begin{flushright}
		\begin{figure}[!h]
			%\includegraphics[scale=0.13]{mittelab}
		\end{figure}
	\end{flushright}
	\begin{center}
		\textbf{\Huge Corso di introduzione all'elettronica}\\	
	\end{center}

	\section{Presentazione} % (fold)
	\label{sec:presentazione}
		In questo corso vogliamo insegnarvi, attraverso numerosi esperimenti pratici, come si comporta la corrente all'interno dei fili, delle resistenze e di qualche componente più complesso (lampadine e led ad esempio). Ovviamente in così poco tempo non vi renderemo degli esperti elettronici ma speriamo di riuscire a fornirvi le basi fondamentali per iniziare a sviluppare autonomamente piccoli circuiti utili 
	% section presentazione (end)

	\section{Programma} % (fold)
	\label{sec:programma}
		\large Prima lezione (3 ore)
		\begin{itemize}
			\item Dall'elettrone alla resistenza: immaginiamoci come funziona (esempio pratico),
			\item viva la Repubblica Italiana: $V=R*I$ la formula che ci permetterà di progettare il nostro circuito,
			\item prima esperienza pratica: tensione e corrente su una resistenza,
			\item resistenze in Serie e Parallelo: vediamo cosa succede se mescoliamo le resistenze,
			\item seconda esperienza pratica: vediamo se i conti che abbiamo fatto sono giusti,
			\item terza esperienza pratica: regoliamo la luminosità di una lampadina con quanto visto fino ad ora.
		\end{itemize}
		\large Seconda lezione (3 ore)
		\begin{itemize}
			\item Il led?
			\item Regole dei nodi e Delle maglie?
		\end{itemize}
	% section programma (end)
	
	\section{Background} % (fold)
	\label{sec:background}
		Concetti matematici elementari: moltipicazione, somma, divisione, sottrazione.
	% section background (end)

\end{document}
